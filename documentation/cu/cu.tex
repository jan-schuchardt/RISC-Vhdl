\Chapter{Das Leitwerk}
Das Leitwerk ist die zentrale Steuereinheit des Prozessors. Es interpretiert
die Befehle und \"uberwacht ihre Ausf\"uhrung durch ALU und MMU. Dazu verwaltet
das Leitwerk den Program-Counter (PC) und das Instruction-Register (IR).

\Section{\"Uberblick}
Im Prozessor wurden die durch das RV32I Base Integer Instruction Set
und durch die RV32M Standard Extension for Integer Multiplication and Division
definierten Befehle implementiert.

Skript das VHDL-code erstellt.

\Section{Conditional Jumps}

\Section{Unconditional Jumps}

\Section{Load Store}

\Section{LUI AUPIC}

\Section{Integer Rechenbefehle}
Der Prozessor wurde auf die in RV32I und RV32M definierten Rechenbefehle
optimiert. Dadurch k\"onnen diese RISC-typischen Befehle mit 3 Takten sehr
schnell ausgef\"uhrt werden.

Division

\Section{Timer und Counter}
Wie in der RISC-V-ISA gefordert gibt es einen Counter, der die Anzahl der
bisher ausgef\"uhrten Befehle speichert. Au\ss{}erdem gibt es einen Timer, der
die Anzahl der vergangenen Takte speichert. Da das FPGA keine Echtzeituhr
bereitstellt, wurde auch hierf\"ur der Taktz\"ahler verwendet.

Ausgelesen werden k\"onnen diese Counter durch die Befehle \Instr{RDINSTRET[H]}\nolinebreak{},
\Instr{RDCYCLE[H]} und \Instr{RDTIME[H]}.

\Statechart{
\node [state] (s1) {MMU: N\"achsten Befehl holen \\ ALU: R[\Ipart{dest}] := \(counter\)};
\node [point,right of=s1] (p2) {};
}{
(s1) -> (p2)
}
