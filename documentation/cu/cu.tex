\Chapter{Das Leitwerk}
Das Leitwerk ist die zentrale Steuereinheit des Prozessors. Es interpretiert
die Befehle und \"uberwacht ihre Ausf\"uhrung durch ALU und MMU. Dazu verwaltet
das Leitwerk den Program-Counter (PC) und das Instruction-Register (IR).

\Section{\"Uberblick}
Im Prozessor wurden die durch das \ISAname{RV32I Base Integer Instruction Set}
und durch die \ISAname{RV32M Standard Extension for Integer Multiplication and
Division} definierten Befehle implementiert. Besondere Aufmerksamkeit wurde
dabei mehr auf Robustheit und weniger auf maximale Geschwindigkeit gelegt.

Um den Implementierungsaufwand bei \"Anderungen von Befehlen zu minimieren
wurde ein Compiler-Skript erstellt, das mehrere Makros bereitstellt, aus denen
dann die Befehle zusammengebaut werden k\"onnen. Das Skript kompiliert dann
eine Eingabe aus diesen Makros in VHDL-Code.

\Section{Conditional Jumps}

\Section{Unconditional Jumps}

\Section{Load Store}

\Section{LUI AUPIC}

\Section{Integer Rechenbefehle}
Der Prozessor wurde auf die in RV32I und RV32M definierten Rechenbefehle
optimiert. Dadurch k\"onnen diese RISC-typischen Befehle mit drei Takten sehr
schnell ausgef\"uhrt werden.

\Subsection{Division}
Da bei der Division unm\"oglich zu garantieren ist, dass diese immer nach drei
Takten beendet ist, muss das Leitwerk hier auf eine Best\"atigung der ALU
warten. Diese sieht vor, dass das Rechenwerk die Leitungen \Vhdl{alu\_data\_in}
auf 0 setzt.


\Section{Timer und Counter}
Wie in der RISC-V-ISA gefordert gibt es einen Counter, der die Anzahl der
bisher ausgef\"uhrten Befehle speichert. Au\ss{}erdem gibt es einen Timer, der
die Anzahl der vergangenen Takte speichert. Da das FPGA keine Echtzeituhr
bereitstellt, wurde auch hierf\"ur der Taktz\"ahler verwendet.

Ausgelesen werden k\"onnen diese Counter durch die Befehle \Instr{RDINSTRET[H]}\nolinebreak{},
\Instr{RDCYCLE[H]} und \Instr{RDTIME[H]}.

\Statemachine{170pt}{
\State{s1}{}{MMU: N\"achsten Befehl holen \\ ALU: R[\Ipart{dest}] := \(counter\)}
\State{s2}{right of=s1}{2 Takte warten}
\State{s3}{right of=s2}{PC := PC + 1 \\ IR := Neuer Befehl}
}{
\Edge{->}{s1}{s2}{}
\Edge{->}{s2}{s3}{\Vhdl{mmu\_ack\_in='1'}}
}
